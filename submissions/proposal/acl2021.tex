%
% File acl2021.tex
%
%% Based on the style files for EMNLP 2020, which were
%% Based on the style files for ACL 2020, which were
%% Based on the style files for ACL 2018, NAACL 2018/19, which were
%% Based on the style files for ACL-2015, with some improvements
%%  taken from the NAACL-2016 style
%% Based on the style files for ACL-2014, which were, in turn,
%% based on ACL-2013, ACL-2012, ACL-2011, ACL-2010, ACL-IJCNLP-2009,
%% EACL-2009, IJCNLP-2008...
%% Based on the style files for EACL 2006 by 
%%e.agirre@ehu.es or Sergi.Balari@uab.es
%% and that of ACL 08 by Joakim Nivre and Noah Smith

\documentclass[11pt,a4paper]{article}
\usepackage[hyperref]{acl2021}
\usepackage{times}
\usepackage{latexsym}
\renewcommand{\UrlFont}{\ttfamily\small}

% This is not strictly necessary, and may be commented out,
% but it will improve the layout of the manuscript,
% and will typically save some space.
\usepackage{microtype}

\aclfinalcopy % Uncomment this line for the final submission
%\def\aclpaperid{***} %  Enter the acl Paper ID here

%\setlength\titlebox{5cm}
% You can expand the titlebox if you need extra space
% to show all the authors. Please do not make the titlebox
% smaller than 5cm (the original size); we will check this
% in the camera-ready version and ask you to change it back.

% Content lightly modified from original work by Jesse Dodge and Noah Smith


\newcommand\BibTeX{B\textsc{ib}\TeX}

\title{CS598 DL4HC Reproducibility Project Proposal}

\author{Michael Miller and Kurt Tuohy \\
  \texttt{\{msmille3, ktuohy\}@illinois.edu}
  \\[2em]
  Presentation link: n/a\url{} \\
  Code link: \url{https://github.com/mich1eal/cs598_dl4hc}} 

\begin{document}
\maketitle

% All sections are mandatory.
% Keep in mind that your page limit is 8, excluding references.
% For specific grading rubrics, please see the project instruction.

\section{Paper ID 41: Symptom Similarity Analysis}
\subsection{General Problem}
\cite{zhang_2019}
What is the general problem this work is trying to do? We are not asking for the specific approach, that’s requested below. An example of a general problem is ‘mortality prediction.’ An example of a specific approach is ‘using recurrent neural network and attention mechanism.’ Do not copy the description in the paper – use your own rewording.

\subsection{Specific Approach}
What is the new specific approach being taken in this work, and what is interesting or innovative about it, in your opinion?

\subsection{Specific Hypotheses to Verify}
What are the specific hypotheses from the paper that you plan to verify in your reproduction study?

\subsection{Additional Ablations}
What are the additional ablations you plan to do, and why are they interesting?

\subsection{Data Access}
State how you are assured that you have access to the appropriate data.

\subsection{Computational Feasibility}
Discuss the computational feasibility of your proposed work – make an argument that the reproduction will be feasible.

\subsection{Code Reuse}
State whether you will re-use existing code (and provide a link to that code base) or whether you will implement yourself.

\section{Paper ID 68: Multilevel Representation Learning}
\subsection{General Problem}
\cite{sohn_2020}
What is the general problem this work is trying to do? We are not asking for the specific approach, that’s requested below. An example of a general problem is ‘mortality prediction.’ An example of a specific approach is ‘using recurrent neural network and attention mechanism.’ Do not copy the description in the paper – use your own rewording.

\subsection{Specific Approach}
What is the new specific approach being taken in this work, and what is interesting or innovative about it, in your opinion?

\subsection{Specific Hypotheses to Verify}
What are the specific hypotheses from the paper that you plan to verify in your reproduction study?

\subsection{Additional Ablations}
What are the additional ablations you plan to do, and why are they interesting?

\subsection{Data Access}
State how you are assured that you have access to the appropriate data.

\subsection{Computational Feasibility}
Discuss the computational feasibility of your proposed work – make an argument that the reproduction will be feasible.

\subsection{Code Reuse}
State whether you will re-use existing code (and provide a link to that code base) or whether you will implement yourself.

\section{Paper ID 117: NLP for Cognitive Therapy}
\subsection{General Problem}
\cite{burger_2021}
What is the general problem this work is trying to do? We are not asking for the specific approach, that’s requested below. An example of a general problem is ‘mortality prediction.’ An example of a specific approach is ‘using recurrent neural network and attention mechanism.’ Do not copy the description in the paper – use your own rewording.

\subsection{Specific Approach}
What is the new specific approach being taken in this work, and what is interesting or innovative about it, in your opinion?

\subsection{Specific Hypotheses to Verify}
What are the specific hypotheses from the paper that you plan to verify in your reproduction study?

\subsection{Additional Ablations}
What are the additional ablations you plan to do, and why are they interesting?

\subsection{Data Access}
State how you are assured that you have access to the appropriate data.

\subsection{Computational Feasibility}
Discuss the computational feasibility of your proposed work – make an argument that the reproduction will be feasible.

\subsection{Code Reuse}
State whether you will re-use existing code (and provide a link to that code base) or whether you will implement yourself.


\bibliographystyle{acl_natbib}
\bibliography{acl2021}

%\appendix



\end{document}
