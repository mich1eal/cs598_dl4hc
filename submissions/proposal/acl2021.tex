%
% File acl2021.tex
%
%% Based on the style files for EMNLP 2020, which were
%% Based on the style files for ACL 2020, which were
%% Based on the style files for ACL 2018, NAACL 2018/19, which were
%% Based on the style files for ACL-2015, with some improvements
%%  taken from the NAACL-2016 style
%% Based on the style files for ACL-2014, which were, in turn,
%% based on ACL-2013, ACL-2012, ACL-2011, ACL-2010, ACL-IJCNLP-2009,
%% EACL-2009, IJCNLP-2008...
%% Based on the style files for EACL 2006 by 
%%e.agirre@ehu.es or Sergi.Balari@uab.es
%% and that of ACL 08 by Joakim Nivre and Noah Smith

\documentclass[11pt,a4paper]{article}
\usepackage[hyperref]{acl2021}
\usepackage{times}
\usepackage{latexsym}
\renewcommand{\UrlFont}{\ttfamily\small}

% This is not strictly necessary, and may be commented out,
% but it will improve the layout of the manuscript,
% and will typically save some space.
\usepackage{microtype}

\aclfinalcopy % Uncomment this line for the final submission
%\def\aclpaperid{***} %  Enter the acl Paper ID here

%\setlength\titlebox{5cm}
% You can expand the titlebox if you need extra space
% to show all the authors. Please do not make the titlebox
% smaller than 5cm (the original size); we will check this
% in the camera-ready version and ask you to change it back.

% Content lightly modified from original work by Jesse Dodge and Noah Smith


\newcommand\BibTeX{B\textsc{ib}\TeX}

\title{CS598 DL4HC Reproducibility Project Proposal}

\author{Michael Miller and Kurt Tuohy \\
  \texttt{\{msmille3, ktuohy\}@illinois.edu}
  \\[2em]
  Presentation link: n/a\url{} \\
  Code link: \url{https://github.com/mich1eal/cs598_dl4hc}} 

\begin{document}
\maketitle

% All sections are mandatory.
% Keep in mind that your page limit is 8, excluding references.
% For specific grading rubrics, please see the project instruction.

\section{Paper ID 41: Symptom Similarity Analysis}
\subsection{General Problem}
\cite{zhang_2019}
What is the general problem this work is trying to do? We are not asking for the specific approach, that’s requested below. An example of a general problem is ‘mortality prediction.’ An example of a specific approach is ‘using recurrent neural network and attention mechanism.’ Do not copy the description in the paper – use your own rewording.

Rate the similarity of pairs of sentences. The idea is to aid prediction of disease by automatically identifying similarities in descriptions of patient symptoms. However, the paper does not use healthcare-related text.

\subsection{Specific Approach}
What is the new specific approach being taken in this work, and what is interesting or innovative about it, in your opinion?

The authors preprocess each sentence by extracting its "trunk" -- the subject, predicate and object. Sentence similarity may identified more reliably by using these trunks.

\subsection{Specific Hypotheses to Verify}
What are the specific hypotheses from the paper that you plan to verify in your reproduction study?

We would reproduce experiment 1. This creates and tests a classifier using a dataset of human-labeled sentence similarities. Experiment 1 compares the authors' method to a baseline plus the methods in four related papers.

\subsection{Additional Ablations}
What are the additional ablations you plan to do, and why are they interesting?

Possible Ablations
\begin{itemize}
  \item Incorporate synonyms, antonyms, and other word relationships into the model to gauge how this affects ratings of sentence similarities. The paper doesn't use healthcare data, so normal English synonyms could be a proxy for gauging whether the model's performance on healthcare data would improve by incorporating medical ontologies. One potential source of synonyms is the \href{https://www.kaggle.com/datasets/duketemon/wordnet-synonyms}{WordNet synonyms dataset} on Kaggle.
\end{itemize}

\subsection{Data Access}
State how you are assured that you have access to the appropriate data.

The training and testing datasets are public: the \href{https://www.microsoft.com/en-us/download/details.aspx?id=52398}{Microsoft Research Paraphrase Corpus} (MSRP) and the SemEval \href{https://github.com/brmson/dataset-sts/tree/master/data/sts/semeval-sts}{Semantic Text Similarity} (STS) datasets.

\subsection{Computational Feasibility}
Discuss the computational feasibility of your proposed work – make an argument that the reproduction will be feasible.

The training dataset is small: 4,076 pairs of sentences. The authors use the \href{https://nlp.stanford.edu/software/lex-parser.shtml}{Stanford Parser} to extract sentence trunks, and this parser is publicly available. Finally, the authors' model uses only a small CNN, with one convolution layer and one pooling layer.

\subsection{Code Reuse}
State whether you will re-use existing code (and provide a link to that code base) or whether you will implement yourself.

To date, the authors have not responded to our request for code. We would implement these methods ourselves.

\section{Paper ID 68: Multilevel Representation Learning}
\subsection{General Problem}
\cite{sohn_2020}
What is the general problem this work is trying to do? We are not asking for the specific approach, that’s requested below. An example of a general problem is ‘mortality prediction.’ An example of a specific approach is ‘using recurrent neural network and attention mechanism.’ Do not copy the description in the paper – use your own rewording.

\subsection{Specific Approach}
What is the new specific approach being taken in this work, and what is interesting or innovative about it, in your opinion?

\subsection{Specific Hypotheses to Verify}
What are the specific hypotheses from the paper that you plan to verify in your reproduction study?

\subsection{Additional Ablations}
What are the additional ablations you plan to do, and why are they interesting?

\subsection{Data Access}
State how you are assured that you have access to the appropriate data.

\subsection{Computational Feasibility}
Discuss the computational feasibility of your proposed work – make an argument that the reproduction will be feasible.

\subsection{Code Reuse}
State whether you will re-use existing code (and provide a link to that code base) or whether you will implement yourself.

\section{Paper ID 117: NLP for Cognitive Therapy}
\subsection{General Problem}
\cite{burger_2021}
What is the general problem this work is trying to do? We are not asking for the specific approach, that’s requested below. An example of a general problem is ‘mortality prediction.’ An example of a specific approach is ‘using recurrent neural network and attention mechanism.’ Do not copy the description in the paper – use your own rewording.

\subsection{Specific Approach}
What is the new specific approach being taken in this work, and what is interesting or innovative about it, in your opinion?

\subsection{Specific Hypotheses to Verify}
What are the specific hypotheses from the paper that you plan to verify in your reproduction study?

\subsection{Additional Ablations}
What are the additional ablations you plan to do, and why are they interesting?

\subsection{Data Access}
State how you are assured that you have access to the appropriate data.

\subsection{Computational Feasibility}
Discuss the computational feasibility of your proposed work – make an argument that the reproduction will be feasible.

\subsection{Code Reuse}
State whether you will re-use existing code (and provide a link to that code base) or whether you will implement yourself.


\bibliographystyle{acl_natbib}
\bibliography{acl2021}

%\appendix



\end{document}
